\documentclass[12pt]{article}

\setlength{\parindent}{0pt}
\setlength{\parskip}{1ex}

\begin{document}

\section{Notation}
\begin{itemize}
    \item $t_n$: date step, $n = 1, \cdots, N$
    \item $m$: item (e.g.\ contract number), $m = 1, \cdots, M$
    \item $s(m,t_n)$: a series of marks. 
    \item If $t_{n}$ is a roll date, 
    \begin{itemize}
        \item $s(m,t_{n-1})$ becomes $s(m-1, t_n)$
        \item $s(1,t_{n-1})$ rolls off, i.e.\ disappears
        \item $s(M, t_{n})$ appears as a new item
    \end{itemize}
    \item $q(m,t_n)$: adjusted marks
    \item $\alpha(m, t)$: drift slope calculated for the $m$-th item as of $t$
    \item $T(m,t)$: the roll-off (i.e.\ maturity) date for the $m$-th item as of $t$
    \item $\phi(m,j)$: last drift slope tracked for the $m$-th item over the $j$-th rolling period $[r_j, r_{j+1})$
    \item $t' - t$: year-fraction from $t$ to $t'$ calculated using the Act 365 convention. 

\end{itemize}

\section{Drift Adjustment}

\subsection{Remarking Criteria}

$t_n$ is a remarked date for the $m$-th item if
\begin{itemize}
    \item if $t_n$ is not a roll date: 
    \begin{equation}
        \left|\frac{s(m, t_n) - s(m, t_{n-1})}{s(m, t_{n-1})} \right| > \epsilon 
    \end{equation}
    \item if $t_n$ is a roll date:
    \begin{equation}
        \left|\frac{s(m, t_n) - s(m + 1, t_{n-1})}{s(m + 1, t_{n-1})} \right| > \epsilon 
    \end{equation}
\end{itemize}
for some $\epsilon > 0$. 


\subsection{Slope Function}

Given $\{s(m,t)\}_{m}$, the $m$-th slope $\alpha(m, t)$ is calculated by
\begin{eqnarray}
    \alpha(1, t) & = & \frac{s(1,t)}{T(1, t) - t} \\
    \alpha(m, t) & = & \frac{s(m,t) - s(m-1,t)}{T(m,t) - T(m-1,t)} \textrm{ for } m = 2, \cdots, M
\end{eqnarray}

{\bf Note}: For standardised products like Futures, we may assume $T(m, t) - T(m-1, t) = 90/365$ for simplicity. 

{\bf Motivation}: Consider two marks: $s(m,t)$ and $s(m-1,t)$.
After time is elapsed by $\Delta := (T(m,t) - T(m-1,t))$, the time-to-roll-off of the first item becomes $(T(m-1,t)-t)$, which is the current time-to-roll-off of the second item. 

Therefore, the natural adjustment slope is the one makes
$$q(m, t + \Delta) = s(m-1, t)$$
if no remarks occurs for all future time steps. 




\subsection{Algorithm}

Consider a rolling period $I_j = \{t_n| t_n \in [r_j, r_{j+1}]\}$ where $r_j$ and $r_{j+1}$ are two adjacent rolling dates. Assume that $t_0 \in I_0$ and $t_N \in I_J$. 

For each rolling period $j$ starting from $0$ to $J$:
\begin{enumerate}
    \item Initialisation: Let $t^j$ be the first date in the period, i.e.\ $t^j := \max\{t_0, r_j\}$ 
    \begin{itemize}
        \item If $j = 0$ or $t^j$ is a remark date, simply set 
        \begin{itemize}
            \item $q(m,t^j) = s(m, t^j)$
            \item $\phi(m, j) = \alpha(m, t^j)$
        \end{itemize}
        for $m = 1, \cdots, M$.
        \item Otherwise, $j > 0$ and $t^j$ is not a remark date. 
        
        For $m = 1, \cdots M-1$, set
        \begin{itemize}
            \item $\phi(m, j) = \phi(m+1, j-1)$, i.e.\ carry over the slope from the previous period
            \item $q(m,t^j) = q(m + 1, p(t^j)) - \phi(m,j) (t^j - p(t^j))$
        \end{itemize}
        where $p(t)$ is the previous time step in $\{t_n\}$. 

        For $m = M$, borrowing from $m = M-1$, set
        \begin{itemize}
            \item $\phi(M, j) = \phi(M-1, j)$, i.e. the slope is carried over. 
            \item $q(M,t^j) = s(M, t^j) + [q(M-1,t^j) - s(M-1,t^j)]$
        \end{itemize}

    \end{itemize}

    \item For each $t_n$ with $t^j < t_n < r_{j+1}$: 
    \begin{itemize}
        \item If $t_n$ is a remarked date, set
        \begin{itemize}
            \item $q(m,t_n) = s(m, t_n)$
            \item $\phi(m, j) = \alpha(m, t_n)$
        \end{itemize}
        for $m = 1, \cdots, M$
        \item Otherwise i.e. not remarked

        \begin{itemize}
            \item $q(m,t_n) = q(m, t_{n-1}) - \phi(m,j) (t_{n} - t_{n-1})$
        
            \item $\phi(m, j)$: no update
        \end{itemize}
        for $m = 1, \cdots, M$


            
    \end{itemize}
\end{enumerate}

\end{document}$