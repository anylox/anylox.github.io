\documentclass[12pt]{article}

\setlength{\parindent}{0pt}
\setlength{\parskip}{1ex}
\usepackage{url}
\newcommand{\glb}{\mathrm{glb}}

\begin{document}

\title{On Recovery Rate Model per DRC}

\section{Reflection}


{\em What didn't go right}: From the model, we have $G^i$, the systematic factor for issuer $i$. At the same time, we have CDS-based times series $\{G^i_n\}$ used to calibrate the default correlation structures. The {\em mistakes} we had in past versions are
\begin{itemize}
    \item treating the CDS-based time series as some data that can reflect the economical condition. The quantities are the changes of CDS levels. One could have 
    \item by intersecting the CDS data with the Moody's data, we had to ignore a large amount of data prior to 2005.  
    \item missing the point that the factor variable $G$ is an abstract variable. It can be interpreted differently in the recovery rate model calibration. In particular, I was not free to think that we can separate the correlation model and recovery model. In other words, 
    \begin{itemize}
        \item CDS data are {\em just} for correlation model
        \item Moody's data are {\em just} for recovery rate model
    \end{itemize}
\end{itemize}


\section{Recollection}


Regulatory {\em Guidelines on LGD Estimates}: 
\url{https://eba.europa.eu/sites/default/documents/files/documents/10180/2551996/f892da33-5cb2-44f8-ae5d-68251b9bab8f/Final%20Report%20on%20Guidelines%20on%20LGD%20estimates%20under%20downturn%20conditions.pdf?retry=1}

\subsection{On LGDs}

\begin{itemize}
    \item Recall that we are supposed to consume the LGDs coming from Credit. 
    \begin{itemize}
        \item For sovereigns, where the firm have the largest exposures, they seem to calculate LGDs as part of IRB, in fact,{\em downturn} LGDs.
        \item For large corps, they have both through-the-cycle and downturn LGDs.
    \end{itemize}
    \item As JK clearly articulated, the fall-back LGD method described in the current methodology would lead to closer to downturn LGDs. This is because there would be a number number of default positions during stressed periods, so taking the average over a long term means that the estimate is biased to the recovery rates from stressed periods. 
    \item The project team proposes to use the F-IRB method to provide fall-back LGDs. Interestingly, the LGD for typical unsecured senior positions under F-IRB is 40\%. This is much lower than the standard 40\% LGD level used in other areas such as CDS valuations. 
    
    If 60\% is not a mistake (despite it has been the case for many years), it is certainly not a downturn LGDs. 

    We have yet to find out whether the project team has spoken to Credit on F-IRB in depth. 
\end{itemize}

\subsection{On PDs}

\begin{itemize}
    \item PDs should be estimated per issuer based on its credit situation {\em as of now}. Indeed, there is no concept of downturn PDs in the regulatory guide. 
    \item The fall-back PD method in the current methodology is calculated by averaging the annual default ratios over a long period by rating. The expectation is that the rating information for each issuer is fully up-to-dated to reflect the current credit quality. 
\end{itemize}



\subsection{On Moody's data}

We can derive the following quantities from the Moody's default historical data from circa 1900:
\begin{itemize}
    \item $\{p_n\}$: 
    \begin{itemize}
        \item annual default ratio (DR) = (num of defaults)/(num of issuers)
        \item varying between 0.5\% and 5.5\%
    \end{itemize}
    \item $\{r_n\}$: 
    \begin{itemize}
        \item anuual average recovery rates of those defaulted names
        \item varying between 20\% and 70\%. 
    \end{itemize}
\end{itemize}
$p_n$ and $r_n$ are visibly opposite to each other. When $p_n$ is high, $r_n$ is low, and vice versa. JK produced a nice time series plot and this should be included in the methodology doc. 

Notes:
\begin{itemize}
    \item $p_n$ can be thought to represent the economic condition. Higher it is, more stressed the economy is. 
    \item Whilst $r_n$ is inversely related to $p_n$, it is not perfectly. So, we would have something like
    \begin{equation}
        r \sim a p + b + \sigma \epsilon
    \end{equation}
    where $a$ would be a negative number. 
    \item At the issuer level, we have to consider an additional idiosyncratic factor as well. 
    \begin{equation}
        r^i \sim ap + b + \sigma \epsilon + \eta^i \epsilon^i
    \end{equation}
\end{itemize}

    





\end{document}